\documentclass[a4paper]{article}
\usepackage{lipsum}
\usepackage{url}
\usepackage{graphicx}
\usepackage{listings}
\usepackage{indentfirst}
\usepackage{enumerate}
\usepackage{multicol}
\lstset{language=Haskell}
\usepackage[margin=2cm]{geometry}
\graphicspath{ {images/} }
\renewcommand{\familydefault}{\sfdefault}

\title{COMP4075/G54RFP Coursework Part III}
\date{16\textsuperscript{th} January 2019}
\author{Benjamin Charlton --- psybc3 --- 4262648}

\usepackage{fancyhdr}

\pagestyle{fancy}
\fancyhf{}
\lhead{Benjamin Charlton | psybc3 | 4262648}
\rhead{G54RFP --- Part III Project}
\cfoot{\thepage}

\begin{document}

% TODO
% Around 5 pages or 1500 words long
%   excluding large code fragments, pictures, and any appendices
% README file in root
%   brief overview of the source code hierarchy
%   making clear what was from scratch - IE all of it

\maketitle

\section{Project Overview}
\subsection{Motivation}
The original basis for this project comes from a series of lab exercises from the G52AIM module, Artificial Intelligence Methods.
The lab exercises involved implementing a variety of AI methods to solve some basic optimisation problems namely MAX-SAT problems.
\par
Many of the methods implemented involved combining smaller functions together to create the desired effect.
This could effectively translated into a functional programming setting.
I thought it would be interesting to try and reimplement these some of these methods in Haskell to see the benefits of the FP paradigm to these AI methods.

\subsection{Technical Background}
% Make the report reasonably self-contained
\subsubsection{MAX-SAT}
MAX-SAT is an optimisation problem

% Implementaion
The MAX-SAT problems were generated by a framework given in the original courseworks, this was via a java file that could be imported.

\subsection{Aim of the Project}

\section{What I did --- Needs another name}
Discussion of the implementation, justifying key decisions and highlighting
and explaining particularly interesting aspects, illustrating with
excerpts from the developed code where appropriate.

\section{Learnt stuff --- Needs another name}
A section reflecting upon what was learned from the project and your
thoughts around the project topic from a real-world programming perspective.

\end{document}
