\documentclass[a4paper]{article}
\usepackage{lipsum}
\usepackage{url}
\usepackage{graphicx}
\usepackage{listings}
\usepackage{indentfirst}
\usepackage{enumerate}
\usepackage{multicol}
\usepackage{enumitem}
\lstset{language=Haskell}
\usepackage[margin=2cm]{geometry}
\graphicspath{ {images/} }
\renewcommand{\familydefault}{\sfdefault}

\title{COMP4075/G54RFP Coursework Part III\vspace{-6mm}}
\date{\vspace{-3mm}16\textsuperscript{th} January 2019}
\author{Benjamin Charlton --- psybc3 --- 4262648\vspace{-8mm}}

\usepackage{fancyhdr}

\pagestyle{fancy}
\fancyhf{}
\lhead{Benjamin Charlton | psybc3 | 4262648}
\rhead{G54RFP --- Part III Project}
\cfoot{\thepage}

\begin{document}

% TODO
% Around 5 pages or 1500 words long
%   excluding large code fragments, pictures, and any appendices
% README file in root
%   brief overview of the source code hierarchy
%   making clear what was from scratch - IE all of it

\maketitle
\vspace{-12mm}

\section{Project Overview}
\subsection{Motivation}
The original basis for this project comes from a series of lab exercises from the G52AIM module, Artificial Intelligence Methods.
The lab exercises involved implementing a variety of AI methods to solve some basic optimisation problems namely MAX-SAT problems.
\par
Many of the methods implemented involved combining smaller functions together to create the desired effect.
This could effectively translated into a functional programming setting.
I thought it would be interesting to try and reimplement these some of these methods in Haskell to see the benefits of the FP paradigm to these AI methods.

\subsection{Aims of the Project}
The original coursework took place over several lab sessions, incorporating a variety of topics and theoretical questions, as well as originally relying heavily of a java based framework.
Due to this only part of the lab exercises will be looked into and some parts of the java framework will have to be remade in Haskell.
\par
Here is what I intend to create in this project:\vspace*{-4mm}
\begin{multicols}{3}
    \begin{itemize}[noitemsep,nolistsep]
        \item MAX-SAT problem generator
        \item MAX-SAT evaluator
        \item Naive solvers
        \item Hill Climbing solvers
        \item Genetic Algorithm Solver
    \end{itemize}
\end{multicols}\vspace*{-3mm}
While creating this components, real world functional programming techniques will be applied to make this work efficient and maintainable.

\subsection{Technical Background}
\subsubsection{MAX-SAT}
MAX-SAT is an optimisation problem which is NP-Hard.
Given a logic formula in conjunctive normal form, the aim is to maximise the number of clauses which are true after variable assignment.
A logical problem in conjunctive normal form has the following formal grammar.
\begin{lstlisting}
CNF -> (C) | (C) & CNF      L -> V | !V
C   -> L   | L + C          V -> 1 | 2 | 3 | ...
\end{lstlisting}
\paragraph{CNF Formula}\vspace*{-6mm}
A CNF formula is made up of clauses which are logically and'd together.
If each clause in a CNF problem is true then the whole formula will return true.
Although in MAX-SAT we aren't aiming for all clauses to be true only to maximise the number that are true.
\paragraph{Clauses C}\vspace*{-4mm}
Each clause contains a series of literals, which are logically or'd together.
This means so long as any literal can evaluate to true the whole clause will be true.
\paragraph{Literals L}\vspace*{-4mm}
A literal is a variable which is either positive or negative, this is achieved by applying a logical not or not doing so.
\paragraph{Variables V}\vspace*{-4mm}
Finally each literal is assigned a variable which are later used to evaluate a solution.
There can be any number of variables so integers are used here in this example to represent the different variables.
\paragraph{Solution for CNF}
A solution for CNF will involve all of the variables in the problem being given a boolean value.
From here you can evaluate each clause and then find how many clauses evaluate to true.
One benefit of how this problem works is that any solution (granting that all variables required have an assignment) will be a valid solution to the problem.

\subsubsection{Mathematical Optimisation}
The MAX-SAT problem is a form of optimisation problem that can be solved by a optimisation technique.
An optimisation problem means that there is an objective function that you wish to maximise the value of by changing the input.
In the case of MAX-SAT the objective function is the number of clauses that evaluate to true given our solution as input.
The goal of any optimisation problem is to find the solution that gives the best possible value for the objective function.
\par
In MAX-SAT there may be several optimal solutions and it is hard to calculate if you do have an optimal solution.
Apart from the trivial case of all clauses being true there is no guarantee that an optimal solution has been found, there is also no guarantee that all clauses can be true either.
\par
While you could devise a way to generate all possible solutions and evaluate them all to find the optimal, due to the NP-hardness of this problem this method will take none polynomial time to run which is very inefficient.
The field of mathematical optimisation aims to find techniques to more efficiently find optimal or near optimal solutions to these types of problems.
While the techniques used for this project don't guarantee optimal solutions the achieve near optimal solutions that have been shown to produce solutions very close to the true optimum.
Although MAX-SAT doesn't have any immediately obvious applications it is very similar to problems such as scheduling where near optimal solutions may be sufficient so the application of these algorithms have real world importance.

\section{Implementaion}
% Discussion of the implementation, justifying key decisions and highlighting and explaining particularly interesting aspects, illustrating with excerpts from the developed code where appropriate.
In this section the implementation of each component will be discussed including any key decisions and interesting aspects.

\subsection{Code Structure}
The code for this project is split in to 3 main modules CNF, Solvers, and Tests.
The idea with this is it separates the code up based upon function and allows the modules to be shared easily depending on what part a user wants.
This was also a massive benefit when programming as it was clear what functionality each submodule provided making debugging easier.
Splitting everything up also allowed for a high level of code reuse throughout the project.
\par
These features of the Haskell programming language, while not functional and seen in other none functional languages, allow for Haskell to be used like other popular languages used in industry.

\subsection{MAX-SAT problem generator}
In the java framework there was the ability to generate a MAX-SAT problem in a suitable form, before any solvers could exist this component and decisions of how the problem will appear also had to be made.
\subsubsection{Creating new types}
To help define the problem more succinctly new types and data were created in the type system, by the code below.
These types are very similar to the formation of the grammar presented in section 1.3.1 and thus was very simple to implement in Haskell from the grammar.
All of the types also derived show so that it could be used to print out the problems to console.
\lstinputlisting[language=Haskell, firstline=2, lastline=5]{reportIIICode.hs}
\subsubsection{Problem Generation}
While in theory a CNF problem can have clauses of differing sizes, it was chosen to make all the clauses contain the same number of literals.
This restriction means that the generator creates a MAX-kSAT where each clause has exactly k literals.
It should be noted that this restriction only applies to the generation of the problem and isn't used by the solvers meaning that the solvers will work on any MAX-SAT problem not just MAX-kSAT\@.
\par
This allows any MAX-kSAT to be defined by 3 values, v the number of variables, c the number of clauses, and r the number of variables per clause.
From there it was simple to create a series of functions that would generate a problem, which can be found in the CNF.Generator module.
\par
The implementation is broken into the generation of a problem, clause, literal and variable.
For an individual variable a random number is picked between 0 and the number of variables in the problem, it is then used in the literal generation by randomly choosing if it will be negated.
For the problem and clause generation a list of the desired size is created by using replicateM.
As randomness is used this means that the problem generated has to be wrapped up in the IO monad.

\subsection{MAX-SAT evaluator}
The final submodule in the CNF module is the evaluator.
This required defining the type of a solution which was simply a mapping of variable numbers to a boolean value.
\lstinputlisting[language=Haskell, firstline=8, lastline=8]{reportIIICode.hs}
\par
An evaluator would therefore have the following type.
\lstinputlisting[language=Haskell, firstline=9, lastline=9]{reportIIICode.hs}
This is quite trivial to then implement with list comprehensions and pattern matching.
A full implementation can be found in the CNF.Evaluator module.

\subsection{Naive solvers}
3 naive solvers are grouped together in the Solvers.Naive module.
These were created to allow for a quick way to check if the generator/evaluator was working correctly but also served as a comparison point for the other algorithms.
One thing all of these solvers have in common is that they don't take the problem into account when creating the solution just the number of variables needed for it to be valid.
These can all be found in the Solvers.Naive module.
\subsubsection{All True/False}
The most basic of ways to create a valid solution is to make all the variables true or false, which is what these 2 functions did.
While these solutions are valid they often don't produce good results.
\lstinputlisting[language=Haskell, firstline=12, lastline=14]{reportIIICode.hs}

\subsubsection{Random Solution}
The other naive way to create a valid solution is to randomly assign boolean values to variables.
This produces an answer which is valid but again not very good, but it does provide a good starting point for the other methods to iterate from.
\lstinputlisting[language=Haskell, firstline=17, lastline=19]{reportIIICode.hs}

\subsection{Hill Climbing Solvers}
The hill climbing solvers can be found in the Solvers.HillClimbing module.
Hill Climbing algorithms come from a set of local search algorithms, where at any step in the search process you look and neighbours (points in solution space nearby the current solution) to determine your next step.
A downside of these algorithms is that they are extremely susceptible to local maxima, although this doesn't make a large difference in this problem as the local maxima are objectively close to the global maximum.
\par
Hill Climbing algorithms are known as anytime algorithms as they can be forcibly stopped at any time and the current best known solution will be a valid solution.
All of the implementations of hill climbing also exploit the fact that if at any step a better solution can't be found, the algorithm has reached a local maxima and will return that solution.
This allowed for early termination, on average quicker run times.
Most helpful part of this was during testing the loop duration could be set arbitrary large and the function will return much earlier than expected.
\subsubsection{Set Up Function}
All of the hill climbing algorithms are split into 2 steps, a setup function and a recursive looping function.
The set up function works similarly for every hill climbing algorithm by first creating a random solution as a starting point (from Solvers.Naive) and then calling the hill climbing loop.
A random starting point is used as it has been shown that this is a a simple and effective way to get good results in local search algorithms.
\subsubsection{Implemented Stratergies}
Three different strategies for hill climbing were implemented.
Each one used a list of neighbours generated (explained in the next section) to determine what the next step should be.
All of these functions can be found in the Solvers.HillClimbing module.
In the following paragraphs the strategy for which neighbour to pick is explained.
\paragraph{Steepest Ascent}\vspace*{-4mm}
This method takes the neighbour that increases the objective value the most.
This is achieved by simply sorting the list of neighbours and their score and taking the highest scoring neighbour.

\paragraph{Simple}\vspace*{-4mm}
This method evaluates all the neighbours similar to steepest ascent but rather than sorting the neighbours to find the best it takes the first improving neighbour.
This can potentially allow for greater exploration of the search space than steepest ascent.

\paragraph{Stochastic Hill Climbing}\vspace*{-4mm}
This method randomly chooses which neighbour will be the next solution.
To ensure that it improves over time only neighbours that improve are actually selected and others are discarded.
This allows for greater exploration than simple hill climbing and has the benefit that on average less solutions have to be evaluated before picking a new solution.

\subsubsection{Neighbours}
A neighbouring solution was defined by any solution that had a single variable flipped from the current solution.
This meant it was simple to systematically create a list of neighbours using list comprehension.
The flipNthValue can be found in Solvers.Common module as it was helpful to be accessed in other solvers.
\lstinputlisting[language=Haskell, firstline=22, lastline=24]{reportIIICode.hs}

\subsection{Genetic Algorithm Solver}
Genetic algorithms are a population based search method, meaning that a collection of solutions are used together to search for a better solution.
% TODO: Explain more

\subsubsection{General Structure}
Although long the general structure of the GA is rather simple so it can be seen here.
\lstinputlisting[language=Haskell, firstline=27]{reportIIICode.hs}
It is broken into 3 major parts, Initialisation, Main Loop, Child Generation.
This a typical structure of a GA and once this was written out it was trivial to determine the type signatures required to allow this to be chained together.

\paragraph{Initialisation}\vspace*{-4mm}
Initialisation simply made a list of random solutions and then evaluated the entire population by mapping the evaluation function across the list.
\paragraph{Selection}\vspace*{-4mm}
Selection was done stochastically for simplicity.
There are other better known methods but they require more computation.
Stochastic selection proves to be suitable enough for a proof of concept.
\paragraph{Crossover}\vspace*{-4mm}
Uniform crossover was used where each child takes roughly 50\% of the genes from each parent.
This is similar to crossover methods in nature and is easily applicable here as any crossover still gives a valid solution.
\paragraph{Mutation}\vspace*{-4mm}
For mutation a random variable in the child solution is negated, using the same bit flip functions used for hill climbing.
\paragraph{Replacement}\vspace*{-4mm}
Replacement takes the weakest members of the old population and replaces them with the new children.

\section{Conclusions}
While implementing these optimisation methods in Haskell, several functional programming techniques where tried.
Some of these succeeded and others failed, this section aims to discuss the pros and cons of using a functional language to implement optimisation techniques.

\subsection{Parallelism}
I attempted to use the Control.Monad.Parallel module to add parallelism to the project as many parts of a GA such as child generation can happen in parallel.
Using this package provided a simple way to achieve parallelism across the computation which is rather shocking compared to other languages like python where this can be rather tricky.
This is a big plus for using functional languages as it is easy for a compiler to make these adjustments and add large optimisations like parallelism.
\par
Unfortunately in this implementation run times were roughly N times longer than single threaded performance for any N numbers of cores.
This can be attributed to the fact that the computation in tasks like generating a child were very small compared to the added overhead of context switching in parallelism.
While it can be easy to then say that this isn't helpful, due to the simplicity of adding parallelism, it can be tested easily and in harder problems this addition would be very beneficial.

\subsection{Typing and Monads}
A feature of strict typing that would have been helpful was being able to us interchangeable functions so long as they had the same type.
This would be beneficial in the creation of a GA where you are piecing together small functions like mutation method and replacement.
As any crossover method would have the type Solution -> Solution it would be easy to create many crossover methods and try them all.
Unfortunately this wasn't possible as some of these methods involved randomness and others didn't meaning some results where in an IO monad.
\par
This was a big downside of using a strictly typed functional language like Haskell with optimisation methods that rely heavily on randomness to explore the search space.
These random computations had to be wrapped up in an IO monad made some parts trickier than if they weren't in the monad.
\par
Technically speaking the pure functions could be also wrapped in an IO monad arbitrarily to create type compatibility but this solution seems like a hack that is hiding information from the programmer.
\par
While the type system was often helpful, when dealing with randomness it feels like a hinderance making much of the code more complicated than it requires.

\subsection{Inherently Procedural Code}
Due to a mix of how GAs work and the use of monadic code, the end result of the GA was more procedural in style than functional.
This is very hard to get rid of an ultimately means that a lot of the benefits of Haskell aren't being exploited.
\par
This shows that in some cases the functional style (although useable) may not be the best choice for implementing an algorithm.

\subsection{Performance and Maintainability}
The performance of all the implementations was as expected with relatively quick run times, although no formal testing was done.
This shows how functional languages can perform as well as other implementations while providing benefits in the code structure, readability and maintainability.
\par
During development changes were easy to make and keep track of what was going on.
There was also rarely any bugs so long as the program compiled, showing that a functional language should be considered when starting a project.
At any given time it was easy to be confident that the code was running efficiently and correctly which is a large benefit compared to other languages.

\subsection{Closing Remarks}
While some aspects of the project were difficult due to Haskell others were quick and simple to implement.
The confidence boost from be sure the code is correct is very helpful when creating the code.
Adapting this code for use on another problem would also be very simple compared to other languages, which could be helpful in industry.
\par
Overall this project was a joy to work on and a nice insight into the world of AI using functional languages.

\end{document}
