\documentclass[a4paper]{article}
\usepackage{lipsum}
\usepackage{url}
\usepackage{graphicx}
\usepackage{listings}
\usepackage{indentfirst}
\usepackage{enumerate}
\usepackage{multicol}
\usepackage{enumitem}
\lstset{language=Haskell}
\usepackage[margin=2cm]{geometry}
\graphicspath{ {images/} }
\renewcommand{\familydefault}{\sfdefault}

\title{COMP4075/G54RFP Coursework Part III}
\date{16\textsuperscript{th} January 2019}
\author{Benjamin Charlton --- psybc3 --- 4262648}

\usepackage{fancyhdr}

\pagestyle{fancy}
\fancyhf{}
\lhead{Benjamin Charlton | psybc3 | 4262648}
\rhead{G54RFP --- Part III Project}
\cfoot{\thepage}

\begin{document}

% TODO
% Around 5 pages or 1500 words long
%   excluding large code fragments, pictures, and any appendices
% README file in root
%   brief overview of the source code hierarchy
%   making clear what was from scratch - IE all of it

\maketitle

\section{Project Overview}
\subsection{Motivation}
The original basis for this project comes from a series of lab exercises from the G52AIM module, Artificial Intelligence Methods.
The lab exercises involved implementing a variety of AI methods to solve some basic optimisation problems namely MAX-SAT problems.
\par
Many of the methods implemented involved combining smaller functions together to create the desired effect.
This could effectively translated into a functional programming setting.
I thought it would be interesting to try and reimplement these some of these methods in Haskell to see the benefits of the FP paradigm to these AI methods.

\subsection{Technical Background}
%TODO MORE HERE
% Make the report reasonably self-contained
\subsubsection{MAX-SAT}
MAX-SAT is an optimisation problem which is NP-Hard.
Given a logic formula in conjunctive normal form, the aim is to maximise the number of clauses which are true.
To do this variables in the formula must be given a boolean value so that the formula can be evaluated, giving a score based upon how many clauses are true.

\subsubsection{Hill Climbing Algorithms}
Hill Climbing algorithms efficiently explore the search space by incrementally change

\subsubsection{Genetic Algorithms}


\subsection{Aims of the Project}
The original coursework took place over several lab sessions, incorporating a variety of topics and theoretical questions, as well as originally relying heavily of a java based framework.
Due to this only part of the lab exercises will be looked into and some parts of the java framework will have to be remade in Haskell.
\par
Here is what I intend to create in this project:\vspace*{-4mm}
\begin{multicols}{3}
    \begin{itemize}[noitemsep,nolistsep]
        \item MAX-SAT problem generator
        \item MAX-SAT evaluator
        \item Naive solvers
        \item Hill Climbing solvers
        \item Genetic Algorithm Solver
    \end{itemize}
\end{multicols}

\section{What I did --- Needs another name}
Discussion of the implementation, justifying key decisions and highlighting
and explaining particularly interesting aspects, illustrating with
excerpts from the developed code where appropriate.

% Implementaion
The MAX-SAT problems were generated by a framework given in the original courseworks, this was via a java file that could be imported.

\section{Learnt stuff --- Needs another name}
A section reflecting upon what was learned from the project and your
thoughts around the project topic from a real-world programming perspective.

\end{document}
